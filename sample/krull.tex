%!TeX program=xelatex
%!TeX spellcheck=fa
%!TeX root = main.tex

\chapter{کرول}
در این فصل برای آغاز به یادگیری عملی کد نویسی در حوزه جبرجابجایی محاسباتی، گاهی نیاز است که ادبیات جبری خود را با اصطلاحات برنامه‌نویسی بازآرایی کنیم.
\section{بعد کرول}
\آغاز{تعریف}
فرض کنیم
$n\le 1$.
\آغاز{شمارش}
\مورد
یک چندجمله‌ای چون
$f\in R[x_1,\dots,x_n]$
که بصورت
$f=x_1^{\alpha_1}\dots x_n^{\alpha_n}$
باشد، که در آن 
$(\alpha_1,\dots,\alpha_n)\in{\bf N}^n$
باشد، را یک \مهم{جمله} یا \مهم{مضرب توانی} گوییم و مجموعه‌ی همه‌ی جمله‌های 
$R[x_1,\dots,x_n]$
با
${\Bbb T}^n$ 
یا
${\Bbb T}(x_1,\dots,x_n)$
نمایش داده می‌شود.

\مورد
برای جمله‌ای چون
$t=x_1^{\alpha_1}\dots x_n^{\alpha_n}\in{\Bbb T}^n$،
عدد
$\deg(t)=\alpha_1+\dots+\alpha_n$
را \مهم{درجه‌}ی $t$ می‌نامند.

\مورد
نگاشت
$\log:{\Bbb T}^n\to{\bf N}^n$
را که با 
$x_1^{\alpha_1}\dots x_n^{\alpha_n}\mapsto(\alpha_1,\dots,\alpha_n)$
تعریف می‌شود را \مهم{لگاریتم} می‌نامیم.

\مورد
اگر
$r\ge 1$
و
$M=(R[x_1,\dots,x_n])^r$،
همان $R[x_1,\dots,x_n]$-مدول تولید شده توسط پایه‌ی کانونی
$\{e_1,\dots,e_r\}$
باشد، آن‌گاه یک  \مهم{جمله} از $M$، یک عنصر به‌شکل
$te_i$
است، که 
$t\in{\Bbb T}^n$
و
$e_i\in\{e_1,\dots,e_r\}$.
مجموعه‌ی همه‌ی جمله‌های $M$ را با
${\Bbb T}^n\!\left<e_1,\dots,e_r\right>$
یا با
${\Bbb T}^n[x_1,\dots,x_n]\left<e_1,\dots,e_r\right>$
(و گاهی هم اگر خطر سوءبرداشت پیش‌نیاید با
${\Bbb T}^{n,r}$
)
نشان می‌دهیم.
\پایان{شمارش}
\پایان{تعریف}

\آغاز{تعریف}
برای $n\ge1$، فرض کنید 
$f=\sum_{\alpha\in{\bf N}^n}c_\alpha t_\alpha\in R[x_1,\dots,x_n]$
یک چندجمله‌ای باشد، و
$m=\sum_{i=1}^r\sum_{\alpha\in{\bf N}^n}c_{\alpha,i} t_\alpha e_i\in M=(R[x_1,\dots,x_n])^r$.
\آغاز{شمارش}
\مورد
برای هر
$\alpha\in{\bf N}^n$
و 
$1\le i\le r$،
$c_{\alpha,i}$
را \مهم{ضریب} جمله‌ی $t_\alpha e_i$ در $m$ گوییم.

\مورد
مجموعه‌ی
$\{t_\alpha e_i\in {\Bbb T}^{n,r}={\Bbb T}^n\!\left<e_1,\dots,e_r\right>|\; c_{\alpha,i}\ne 0\}$
را \مهم{پشتیبان} $m$ گویند و با
$\Supp(m)$
نشان می‌دهند.

\مورد
اگر 
$f\ne 0$،
عدد 
$\max\{\deg(t_\alpha)\;|\;t_\alpha\in\Supp(f)\}$
را \مهم{درجه}ی $f$ خوانیم و آن‌را با
$\deg(f)$
نشان می‌دهیم.
\پایان{شمارش}
\پایان{تعریف}
