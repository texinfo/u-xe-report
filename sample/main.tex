%!tex program = xelatex
%!tex encoding = UTF-8 Unicode
%!tex spellcheck = fa
\documentclass[doctorate,defaults,
amiri,amsmath,showtutorialintoc,
%lalezarchapter,
times,mathptmx,
%index,listpersons,vojenome,
%powerpointmode,%wide,
%badgeo,
%bookcut,
%alt,
qi,
unicode-math]
{u-xe-report}

\usepackage{endnotes}

\def\mhm#1{\textbf{#1}}
\DeclareMathOperator{\mon}{mon}
\def\dsb#1{\Bbb{#1}}
\DeclareMathOperator{\nil}{nil}
\def\Specmax{\Spec_{\max}}
\def\Specrab{\Spec_{\mathrm{Rab.}}}


\newfontfamily\unicodefontforsymbols{Segoe UI Symbol}

\newfontinstance\xits[ItalicFeatures={FakeSlant=.41667},BoldItalicFeatures={FakeSlant=.41667}]{xits-math.otf}
\newfontinstance\asana{Asana-Math.otf}
\setdigitfont{IRANSans}
\setlatintextfont[Scale=1.315]{Myriad Arabic}

\surname[Ghayour]{غیور نجف‌آباد}

\geroyec[Commutative Algebra]{جبرجابجایی}
%\usepackage{unicode-math}
%\setmathfont{xits-math.otf}
%\usepackage{mathptmx}
%\usepackage{newunicodechar}
%\newunicode{22F0}{\adots}}


\mathsubjprimary[۱۳پ۱۰]{13P10}
\mathsubjsecondary[۶۸ئو۳۰]{68W30}
%\mathsubjsecondary[۱۴ق۰۱]{14Q01}
\mathsubjsecondary[۱۶ز۰۵]{16Z05}
\title[Computation of Grobner Basis]
{محاسبه پایه‌ی گروبنر}
\supervisor[Omid-Ali Shahny Karamzadeh]
{امیدعلی شهنی کرم‌زاده}
\advisor[Mehrdad Namdari]{مهرداد نامداری}
\supervisor[Maryam Davoodian]{مریم داودیان}
\nom[Omid]{امید}
\nomexonevodegi[Ghayour]{غیور}
\torix[2017]{۱۳۹۶}
\torixenegorec[Sunday, March 8, 2015]{یکشنبه ۱۷ اسفند ۱۳۹۳}	
\kelidvojeho
[Ring of Polynomials over a Field, Ideals, Noetherian dimension, Artinian Modules, Polynomoids, \& Length]
{ حلقه‌ی چند‌جمله‌ای‌ها روی یک میدان، ایده‌آل‌ها، بعد نوتری، مدول‌های آرتینی، چندجمله‌ای‌وارها، و طول}

\tcekide
[In this work we studied on comptation, computability, and implimentation of computational algorithms for ideals of polynomials over a field and commutative modules.]
{این پایان‌نامه، به بحث در مورد محاسبه و محاسبه پذیری و پیاده سازی الگوریتم‌های محاسباتی برای ایدآل‌های چندجمله‌ای‌ها بر روی یک میدان و مدول‌های روی حلقه‌های جابجایی پرداخته‌ایم. }

\seposguzorifile{seposguzori}

\newfontinstance\greek[Scale=1.25]{GFS Didot Classic}

\muqaddameqoute
[-- \شخص{}{افلاطون}(نوشته‌ی بالای درب آکادمیا)]
{{\lr{{\greek ἀγεωμέτρητος μηδεὶς εἰσίτω}}}
\\{\nastaliq
 هر کس که هندسه نمی‌داند  وارد  نشود}}
\muqaddamefile{muqaddame}
\muqaddametitle{پیش‌‌‌‌گفتار}
%%\taqdimiye[تقدیم به]
%%{
%%\begin{tabularx}{\textwidth}{lX}
%%همسرم {\unicodefontforsymbols ⚘}&که دلیل اصلی نگارش این رساله بود. \\
%%مادرم {\unicodefontforsymbols ❀}&که دوست داشت دکتر شوم، البته، \\
%%& دوست داشت پزشک یا دکترمهندس شوم! {\unicodefontforsymbols ☺}\\ 
%%\\
%%پدرم و پسرم&مردان عزیزی که آن‌ها را آینه‌ی خود می‌دانم!\\
%%خواهرم و دخترم&که خنده‌های شیرین‌شان همیشه انگیزه‌بخش بوده.\\
%%\\
%%پروفسورکرم‌زاده&که افکارش مسیر ادامه‌ی تحصیلم را تغییر داد! \\
%%\\
%%\end{tabularx}
%%
%%\noindent
%%افلاطون، ارسطو، فیثاغورس، اقلیدس، خوارزمی، خیام، اویلر، اردیش، مندل‌برات، پرلمان، آسیموف، شهریاری، و کنوث
%%
%%\hfill --- که کارهایشان مرا شیفته‌ی ریاضی کرد. \\
%% 
%%
%% 
%% و معلمانی که نامشان را فراموش نمی‌کنم.
%% 
%% ~\hfill
%%\begin{minipage}{6cm}
%%میسیز وودز -- از دبستان
%%
%%آقای باورصاد -- از راهنمایی
%%
%%آقای باسردویی -- از دبیرستان
%%
%%پروفسور معتمدی -- از کارشناسی
%%\end{minipage}
%%\hfill~\\[2mm]
%%
%%\noindent 
%%مادربزرگ‌هایم و\\
%%\indent
%% پدربزرگم ({خواجه ابوالمحمد حافظی بیرگانی})
%% 
%%~\hfill  — که برای من الگویی بوده است، تا هرگز دست از مطالعه بر ندارم.
%%
%%~\hfill ~ — روحشان شاد!
%%
%%}

%\includeonly{dovwdiyon}
%!tex program=xelatex
%!tex root=main.tex
\def\bul{$\bullet$}
\newcommand\butzero{\setminus\zero}
\DeclareMathOperator{\ideal}{Ideal}
\DeclareMathOperator{\spoly}{S-poly}
\DeclareMathOperator{\co}{co}
\DeclareMathOperator{\gdim}{G-dim}%{dim^G\kern -.125em}
\DeclareMathOperator{\kdim}{k-dim}%{dim^K\kern -.125em}
\DeclareMathOperator{\ndim}{n-dim}%{dim^N\kern -.125em}
\DeclareMathOperator{\RF}{RF}
\DeclareMathOperator{\Spec}{Spec}
\newcommand{\s}{$\mathrm{S}$}
\renewcommand{\S}{$\mathbf{\color{myboldcolor}S}$}
\newcommand{\rf}[1]{\overline{#1}}
\newcommand{\Rf}[2]{\rf{\RF(#1,#2)}^{#1}}
\DeclareMathOperator\Var{\mathcal{V}}
\newcommand{\CoCoA}{{\tt CoCoA}}
\newcommand\V{\Var}
\DeclareMathOperator\LeadTerm{LT}
\DeclareMathOperator\LeadCoeff{LC}
\DeclareMathOperator\LeadMonom{LM}
\DeclareMathOperator\NormalForm{NF}
\DeclareMathOperator\Syzygy{Syz}
\DeclareMathOperator\Hamyuqi{\text{\scriptsize\nastaliq می}\!}%{\text{\scriptsize\nastaliq می}\!}%{{\frak G}}
\DeclareMathOperator\Supp{Supp}
\newcommand{\uses}[1]{\xrightarrow{~ #1 ~}}
\newcommand{\used}[1]{\xleftarrow{~ #1 ~}}
\newcommand{\using}[1]{\xleftarrow{~ #1 ~}\kern -1em \to}
\newcommand\LT[1][\sigma]{\LeadTerm_{#1}}
\newcommand\LM[1][\sigma]{\LeadMonom_{#1}}
\newcommand\LC[1][\sigma]{\LeadCoeff_{#1}}
\newcommand\NF[1][\sigma]{\NormalForm_{#1}}
\newcommand\Syz[2][]{\Syzygy_{#1}({#2})}
\newcommand\Hmy[2][\,]{\Hamyuqi_{#1}({#2})}
\newcommand\rel[1][\sigma]{\le_{\tt #1}}
\newcommand\ler[1][\sigma]{\ge_{\tt #1}}
\newcommand\zero{{\{0\}}}
\newcommand\gives[1][~]{\xrightarrow{~#1~}}
\newcommand\givesby[1]{\overset{#1}\longrightarrow}

\newcommand\aSet{}
\def\aSet\{#1|#2\}{\left\{\,#1 \;\middle|\;\allowbreak #2 \,\right\}}
\newcommand\aset{}
\def\aset\{#1|#2\}{\{\,#1 \mid #2 \,\}}
\newcommand\Set[1]{\aSet\{#1\}}
\newcommand\set[1]{\aset\{#1\}}

\DeclareMathOperator{\End}{End}
\DeclareMathOperator{\Ann}{Ann}


\usepackage{lipsum}
\def\maxswsname{پیاده‌سازی}
\begin{document}
%\بخش{واژگان جبرجابجایی}%آلمان
%\setdigitfont[Scale=1.728]{Yas}
\include{sefrgoh2}
%\include{zoriski}
%\include{soyer}
%%%%%%%%%%%%%%
\include{krull}
%\بخش{محاسبات جبری}%ایتالیا
\include{bunyod}
\include{gruybner}
\include{diksun-forsi}	
\include{dovwdiyon}
%\بخش{چالش‌های پایه‌ی گروبنر}%ایران
\appendix % in dastwr baroye in ast ke mucaxxas kunim az indjo be ba'd zamoem darj micavand
%\include{korburd}\include{piyodesozi}\include{behbwdyofte}
%\include{poyon}
%
%\include{avvalin}
%\include{poyon}
%\include{adabiyot}
%\include{torix}
%\include{darbore}
\makeatletter
\if@badgeoopted
\baselineskip=9mm
\fi
\makeatother
%!tex program=xelatex
%!tex root=main.tex
% مراجع خود را در این قسمت وارد کنید
\begin{thebibliography}{47}
\addcontentsline{toc}{chapter}{\numberline{~}مراجع}
% چنانچه مرجع فارسی هم دارید باید دستور زیر را فعال کرده و مراجع فارسی خود را بعد از این دستور وارد کنید
%\persian
%\bibitem{11} 
%آذرپناه, فریبرز, نخستین درس در توپولوژی, انتشارات دانشگاه شهید چمران اهواز, %1382
%\bibitem
% بهبودی، محمود، پایان نامه دکتری، انتشارات دانشگاه شهید چمران اهواز, 1382
\latin
\LTRbibitems
%\baselineskip=3.5ex

\bibitem{a-r}
Albu, T., Rizvi, S. (2001). Chain conditions on quotient finite dimensional mod-
ules.Comm. Algebra 29 (5): 1909-1928

\bibitem{al-sm-2}
Albu, T., and Smith, P. F., {\it Localization of modular lattices, Krull dimension, and the Hopkins-Levitzki Theorem (I)},
Math. Proc Cambridge Philos Soc. 120 (1996), 87-101.

\bibitem{al-sm-3}
Albu, T., and Smith, P. F., {\it Localization of modular lattices, Krull dimension, and the Hopkins-Levitzki Theorem (II)}, Comm. Algebra 25 (1997), 1111-1128.

\bibitem{al-sm}
Albu, T., and Smith, P.F., {\it Dual Krull dimension and duality}, Rocky Mountain J.Math. 29 (1999), 1153-1165.

\bibitem{al-t1p}
Albu, T., and Teply, L., {\it Generalized deviation of posets and modular lattices}, Discrete Math. 214 (2000), 1-19.

\bibitem{al-va}
Albu, T., and Vamos, P., {\it Global Krull dimension and global dual Krull dimension of valuation rings}, Lecture Notes in Pure and Applied Mathematics. Vol 201 (1998), 37-54.


\bibitem{an-fu}
Anderson, F.W., and Fuller, K.R., {\it Rings and categories of modules}, Springer-Verlag, 1992.

\bibitem{ar1}
Armendariz, P., {\it Rings with an almost Noetherian ring of fractions}, Math. Scand. 41 (1977), 15-18

\bibitem{1}
Atiyah, M., MacDonald, I.G., {\it Introduction to Commutative Algebra}, Addison-Wesley, London, 1969.


\bibitem{bass}
Bass, H., {\it Descending chains and the Krull ordinal of commutative rings}, J.
Pure Appl. Algebra 1: 347--360.


\bibitem{b-g}
Bilhan, G., and Gungoroglu, T., {\it w-coatomic modules}, Cank. Uni. J. Eng. 7(2010),No1. 17-24.

\bibitem{b-h}
Bilhan, G., and Hatipoglu, C., {\it Finitely coatomic modules}, Hacet. J. Math. Stat. 36(2007), 37-41. 

\bibitem{b-s}
Bilhan, G., and Smith, P.F., {\it Short modules and almost Noetherian modules}, Math. Scand. 98 (2006), 12-18.

\bibitem{camilo}
 Camillo, V. P., and Zelmanowitz, J., {\em On the dimension of a sum of modules}, Comm. Algebra, 6(1978), no 4, 345-352.

\bibitem{ch}
Chambless, L., {\it N-Dimension and N-critical modules, Application to Artinian
modules}, Comm. Algebra 8 (1980), 1561-1592.

\bibitem{co1}
Cohen, I., {\it Commutative rings with restricted minimum condition}, Duke Math. J. 17 (1950), 27-42.

\bibitem{c-l}
Croisot, R., Lesieur, L., {\it Sur les anneaux premier a gauch}, Ann. sci. E'col. Norm. sup, 79(1959), 161-183.

\bibitem{d-f}
Dauns, J., Fuchs, L., {\it Infinite Goldie dimension}, J. Algebra, 115, (1983), 247-302.

\bibitem{d-k}
Davoudian, M., Karamzadeh, O. A. S., 
{\it Artinian serial modules over commu-
tative (or left Noetherian) rings are at most one step away from being Noetherian.}
Comm. Algebra, to appear.

\bibitem{d-k-sh} 
Davoudian, M., Karamzadeh, O. A. S., Shirali, N., {\it On $\alpha$-short modules}, to appear. 

\bibitem{dung}
Dung, N. V., Huynh, D. V., Smith, P. F., and Wisbauer, R., {\em Extending Modules}, Longman, Harlow, 1990.

\bibitem{fu}
Fuchs, L., {\it Torsion preradical and ascending Loewy series of modules},
J. Reine und Angew. Math. 239 (1970), 169-179.

\bibitem{g1}
Goldie, A.W., {\it The structure of prime rings under ascending chain conditions}, Proc. London Math. soc. VIII(1958), 589-608.

\bibitem{g1}
Goodearl, K.R., {\it Von Neumann ergular rings}, Pitman, san, Francisco, 1979.

\bibitem{g2}
Goodearl, K.R., Warfield, JR. R. B., {\it An Introduction to Noncommutative Noetherian rings}, Cambridge University Press, 1989.

Goodearl, K. R., Zimmermann-Huisgen, B. (1986). Lengths of submodule chain
versus Krull dimension in Non-Noetherian modules. Math. Z. 191: 519-527.

Gordon, R. (1974). Gabriel and Krull dimension, in : Ring Theory (Proceeding of
the Oklahoma Conference), Lecture Notes in Pure and Appl. Math. Vol. 7: Dekker,
NewYork pp. 241-295.

\bibitem{go-ro}
Gordon. R., and Robson, J.C., {\it Krull dimension}, Mem. Amer. Math. Soc. 133, 1973.
 
\bibitem{gu1}
Gungoroglu, G. {\it Coatomic modules}, Far East J. Math. Sci. Special Volume (1998), part  II, 153-162

\bibitem{kh}
Hashemi, J., Karamzadeh, O.A.S., and Shirali, N., {\it Rings over
which the Krull dimension and the Noetherian dimension of all
modules coincide}, Comm. Algebra 37 (2009), 650-662.

\bibitem{h1}
Hein, J., {\it Almost Artinian modules}, Math. Scand. (1979) 198-204.

\bibitem{hi}
Hirano, Y., {\it On rings over which each modules has a maximal submodule}, Comm. Algebra 26 (1998), 3435-3445.

\bibitem{ka1}
Karamzadeh, O.A.S., {\it Noetherian-dimension}, Ph.D. thesis, Exeter, 1974.

Karamzadeh, O.A.S. (1982) When are Artinian modules countable generated? Bull.
Iran. Math. Soc., 9, 171-176.
[14

\bibitem{ka5}
Karamzadeh, O.A.S., and Sajedinejad, A.R., {\it Atomic modules},
Comm. Algebra 2001, 29 (7), 2757-2773.

\bibitem{ka6}
Karamzadeh, O.A.S., and Sajedinejad, A.R., {\it On the Loewy length and the
Noetherian dimension of Artinian modules}, Comm. Algebra 30 (2002), 1077-1084.


\bibitem{ksh}
Karamzadeh, O.A.S., and Shirali, N., {\it On the countablity of Noetherian dimension of Modules}, Comm. Algebra 32 (2004), 4073-4083.


\bibitem{ka3}
Karamzadeh, O.A.S., and Motamedi, M., {\it On $\alpha$-$DICC$ modules},
Comm. Algebra 22 (1994), 1933-1944.

\bibitem{ka4}
Karamzadeh, O.A.S., and Motamedi, M., {\it a-Noetherian and Artinian modules},
Comm. Algebra 23 (1995), 3685-3703.


\bibitem{kas1} 
Kasch, F., {\it Modules and Rings}, London Mathematical Society Monographs, Vol. 17, Academic press, London 1982.


\bibitem{ki}
Kirby, D., {\it Dimension and length for Artinian modules}, Quart.
J. Math. Oxford. 41 (1990), 419-429.

\bibitem{kr}
Krause, G., {\it On fully left bounded left Noetherian rings}, J. Algebra 23 (1972), 88-99.

Krause, G. (1973). Descending chains of submodules and the Krull dimension of
Noetherian modules. J. Pure Appl. Algebra 3: 385-397.

\bibitem{le2}
Lemonnier, B., {\it D\'{e}viation des ensembles et groupes ab\'{e}liens
totalement ordonn\'{e}s}, Bull. Sc. Math. 96 (1972), 289--303.

\bibitem{le3}
Lemonnier, B., {\it Dimension de Krull et codeviation, Application au theorem d'Eakin}, Comm. Algebra 1978, 6, 1647-1665.

\bibitem{m1}
Matlis, E., {\it Modules with descending chain condition}, Trans. Amer. Math. Soc. 98(1960) 459-508.
	
\bibitem{m2}
Matlis, E., {\it Some properties of Noetherian domains of dimension one}, Canad. J. Math. 13 (1961) 569-586.

\bibitem{mc}
McConell, J.C., and Robson, J.C., {\it Noncommutative Noetherian Rings}, Wiley-Interscience, New York, 1987. 

\bibitem{puz}
Puczylowsky, E.R., {\it On the uniform dimension of the radical of a module}, Comm. Algebra 1995, 23(2), 771-776.

\bibitem{r-g}
Rentschler, P., and Gabriel, P., {\it Sur la dimension des anneaux et ensembles ordonn\'{e}s}, C. R. Acad. Sci. Paris 265 (1967), 712-715.

 \bibitem{ro}
Roberts, R.N., {\it Krull-dimension for Artinian Modules over
quasi local commutative Rings}, Quart. J. Math. Oxford. 26(1975),
269-273.

\bibitem{sa1}
Sarath, B., {\it Krull dimension and Noetherianess}, Illinois J. Math., 20(1976), 329-335.

\bibitem{13}
Sharp, R.Y., {\it Steps in Commutative Algebra}, Cambridge University Press, 1990.

\bibitem{Sharp2}
Sharp, R.Y., {\it A method for the study of Artinian modules with an application to asymptotic behavior}, Proceeding of micr-program(Commutative Algebra), Springer-Verlag, (1989), 443-464.

\bibitem{valle}
Valle, A., {\it Goldie dimension of a sum of modules}, Comm. Algebra 1994, 22(4), 1257-1269.

\bibitem{va1}
Vamos, P., {\it The dual of the notion of finitely generated}, J. London. Math. Soc,   43(1968), 643-646. 

\bibitem{w1}
Weakley, W. D., {\it Modules whose proper submodules are finitely generated}, J. Algebra, 84(1983), 189-219.

Wisbauer, R. Foundations of module and ring theory. Gordon and Breach, Philadel-
phia, 1991.

%%%\Persian
%\bibitem{smyth2}
%M.B. Smyth, Topology, S. Abramsky, D.M. Gabbay, T.S.E. Maibaum (Eds.), {\em Handbook of Logic in
%Computer Science}, Vol. 1, Clarendon Press, Oxford, 1992, pp. 641–761.
%\bibitem{topsze}
%F. Topsze, {\em Topology and Measure}, Lecture Notes in Mathematics, Vol. 133, Springer, Berlin, 1970.
\end{thebibliography}


%\bibliographystyle{umid-scu}
%\bibliography{mylibrary}

\end{document}
